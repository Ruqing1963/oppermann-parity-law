% ═══════════════════════════════════════════════════════════════════════
% Oppermann's Parity Law: Quadratic Residue Symmetry Breaking
% in Half-Intervals via the Negation Involution
% Titan Project — Paper X — February 2026
% ═══════════════════════════════════════════════════════════════════════

\documentclass[11pt, a4paper]{article}

\usepackage[top=28mm, bottom=28mm, left=25mm, right=25mm]{geometry}
\usepackage[T1]{fontenc}
\usepackage{amsmath, amssymb, amsthm, mathtools}
\usepackage{mathrsfs}
\usepackage[dvipsnames]{xcolor}
\usepackage{enumitem}
\usepackage{booktabs}
\usepackage{hyperref}
\usepackage{float}

\newtheorem{theorem}{Theorem}[section]
\newtheorem{lemma}[theorem]{Lemma}
\newtheorem{proposition}[theorem]{Proposition}
\newtheorem{corollary}[theorem]{Corollary}
\theoremstyle{definition}
\newtheorem{definition}[theorem]{Definition}
\newtheorem{example}[theorem]{Example}
\theoremstyle{remark}
\newtheorem{remark}[theorem]{Remark}

\newcommand{\Leg}[2]{\left(\frac{#1}{#2}\right)}
\newcommand{\Lhalf}{\mathcal{L}_n}
\newcommand{\Rhalf}{\mathcal{R}_n}

\title{\textbf{Oppermann's Parity Law: \\
Quadratic Residue Symmetry Breaking in \\
Half-Intervals via the Negation Involution}}
\author{Ruqing Chen\\[4pt]
\small GUT Geoservice Inc., Montr\'eal, Canada\\[2pt]
\small \texttt{ruqing@hotmail.com}\\[2pt]
\small Repository: \url{https://github.com/Ruqing1963/oppermann-parity-law}}
\date{February 2026}

\begin{document}

\maketitle

\begin{abstract}
We refine the Spin Asymmetry Theorem of \cite{ChenSpin}
by analyzing the distribution of quadratic residues in the
\emph{left} and \emph{right halves} of Legendre intervals,
motivated by Oppermann's conjecture on primes in half-intervals.
For $n \ge 2$ with $p = 2n - 1$ prime, the interior of
$\mathcal{I}_n = [(n-1)^2, n^2]$ splits at the midpoint
$n(n-1)$ into a left half $\Lhalf$ and a right half $\Rhalf$,
each containing $n - 1$ integers.
We prove that when $n$ is \emph{even}
(equivalently $p \equiv 3 \pmod{4}$),
the right half $\Rhalf$ has \emph{exactly zero}
quadratic residue skewness modulo $p$:
the number of QR classes equals the number of NR classes.
The proof uses the negation involution
$x \mapsto p - x$ on $(\mathbb{Z}/p\mathbb{Z})^*$,
which swaps QR $\leftrightarrow$ NR when $p \equiv 3 \pmod{4}$,
combined with the fact that $\Rhalf$ is closed under this
involution.
Consequently, the global skewness $N^- - N^+ = 1$ from
\cite{ChenSpin} is carried \emph{entirely} by the left half.
The theorem is verified computationally for all
87 qualifying even $n \le 500$.
We also prove a complementary result for odd $n$
($p \equiv 1 \pmod{4}$): the right half skewness is
always even and generically nonzero.
\end{abstract}

\bigskip

% ═══════════════════════════════════════════════════════════════════════
\section{Introduction}\label{sec:intro}
% ═══════════════════════════════════════════════════════════════════════

\subsection{Oppermann's conjecture}

Oppermann's conjecture (1882) strengthens Legendre's conjecture
by asserting that for every $n \ge 2$, there exists a prime in
\emph{each} of the two half-intervals
\[
    (n^2 - n, \, n^2)
    \quad\text{and}\quad
    (n^2, \, n^2 + n).
\]
Equivalently, using our indexing of the Legendre interval
$\mathcal{I}_n = [(n-1)^2, n^2]$, the conjecture demands
primes on \emph{both} sides of the midpoint $n(n-1)$.

In the companion paper \cite{ChenSpin}, we proved the
Spin Asymmetry Theorem: the interior of $\mathcal{I}_n$
(when $p = 2n - 1$ is prime) has exactly one more quadratic
non-residue than quadratic residue modulo $p$.
This global asymmetry was left undissected---it was not known
how the skewness distributes between the two halves.

\subsection{Main result: the Parity Law}

We prove (Theorem~\ref{thm:parity}) that for even $n$,
the global skewness is distributed in the most extreme
possible way: the right half carries \emph{zero} skewness,
and the left half carries \emph{all} of the asymmetry.

The proof reveals a clean algebraic mechanism:
the negation involution $\iota : x \mapsto p - x$ on
$(\mathbb{Z}/p\mathbb{Z})^*$
swaps QR $\leftrightarrow$ NR precisely when $p \equiv 3 \pmod{4}$
(i.e., when $n$ is even),
and the residue classes of $\Rhalf$ are closed under $\iota$.
This forces perfect QR/NR pairing in $\Rhalf$.

\subsection{Context within the Titan Project}

This paper continues the algebraic programme initiated in
\cite{ChenSpin}, building on the conductor rigidity framework
of \cite{ChenConductor, ChenLandau, ChenAP} and the
function field resolution in \cite{ChenFF}.

% ═══════════════════════════════════════════════════════════════════════
\section{Setup and Preliminaries}\label{sec:setup}
% ═══════════════════════════════════════════════════════════════════════

\begin{definition}[Half-intervals]\label{def:halves}
For $n \ge 2$, the \emph{left half} and \emph{right half}
of the Legendre interval $\mathcal{I}_n = [(n-1)^2, n^2]$ are
\begin{align*}
    \Lhalf &= \{(n-1)^2 + 1,\; \ldots,\; n(n-1)\}, \\
    \Rhalf &= \{n(n-1) + 1,\; \ldots,\; n^2 - 1\}.
\end{align*}
Each half has cardinality $n - 1$.
The full interior is
$\mathcal{I}_n^\circ = \Lhalf \cup \Rhalf$.
\end{definition}

\begin{definition}[Local skewness]\label{def:skewness}
For a set $S$ of integers and an odd prime $p$, define
\[
    \mathrm{Skew}_p(S)
    \;=\;
    \#\{x \in S : \Leg{x}{p} = -1\}
    \;-\;
    \#\{x \in S : \Leg{x}{p} = +1\}.
\]
\end{definition}

From \cite{ChenSpin}, we know
$\mathrm{Skew}_p(\mathcal{I}_n^\circ) = 1$
whenever $p = 2n - 1$ is prime.
By additivity:
\begin{equation}\label{eq:global}
    \mathrm{Skew}_p(\Lhalf) + \mathrm{Skew}_p(\Rhalf) = 1.
\end{equation}

% ═══════════════════════════════════════════════════════════════════════
\section{The Anchor Position}\label{sec:anchor}
% ═══════════════════════════════════════════════════════════════════════

By \cite[Lemma~2.4]{ChenSpin}, the interior
$\mathcal{I}_n^\circ$ contains a unique multiple of $p$
(the \emph{anchor}, with Legendre symbol $0$).
Its position determines which half inherits the anchor.

\begin{lemma}[Anchor offset]\label{lem:anchor}
Let $n \ge 2$ with $p = 2n - 1$ prime.
The unique multiple of $p$ in $\mathcal{I}_n^\circ$ is located
at offset
\[
    \delta
    \;=\;
    \begin{cases}
    \dfrac{3n - 2}{2} & \text{if $n$ is even
    (i.e., $p \equiv 3 \pmod{4}$),} \\[8pt]
    \dfrac{n - 1}{2} & \text{if $n$ is odd
    (i.e., $p \equiv 1 \pmod{4}$),}
    \end{cases}
\]
from the left endpoint $(n-1)^2$.
In particular:
\begin{enumerate}[nosep, label=(\roman*)]
\item If $n$ is even, then $\delta \ge n$,
so the anchor lies in $\Rhalf$.
\item If $n$ is odd, then $\delta < n$,
so the anchor lies in $\Lhalf$.
\end{enumerate}
\end{lemma}

\begin{proof}
The anchor $m$ satisfies $m \equiv 0 \pmod{p}$ and
$(n-1)^2 < m < n^2$.
Its offset from $(n-1)^2$ is
$\delta = m - (n-1)^2 \equiv -(n-1)^2 \equiv -n^2 \pmod{p}$
(using $(n-1)^2 \equiv n^2 \pmod{p}$ from \cite{ChenSpin}).

From $4n^2 = (2n-1)(2n+1) + 1$, we get $4n^2 \equiv 1 \pmod{p}$,
so $n^2 \equiv 4^{-1} \pmod{p}$.

\emph{Case $n$ even} ($p \equiv 3 \pmod{4}$):
$4^{-1} \equiv (p+1)/4 \pmod{p}$ since
$4 \cdot (p+1)/4 = p + 1 \equiv 1$.
Thus $\delta \equiv -\tfrac{p+1}{4}
\equiv \tfrac{3p-1}{4}
= \tfrac{3(2n-1)-1}{4}
= \tfrac{3n-2}{2} \pmod{p}$.
Since $n \le (3n-2)/2 \le 2n-2$ for $n \ge 2$,
$\delta$ falls in the right half.

\emph{Case $n$ odd} ($p \equiv 1 \pmod{4}$):
$4^{-1} \equiv (3p+1)/4 \pmod{p}$ since
$4 \cdot (3p+1)/4 = 3p + 1 \equiv 1$.
Thus $\delta \equiv -\tfrac{3p+1}{4}
\equiv \tfrac{p-1}{4}
= \tfrac{n-1}{2} \pmod{p}$.
Since $1 \le (n-1)/2 \le n-1$ for $n \ge 3$,
$\delta$ falls in the left half.
\end{proof}

% ═══════════════════════════════════════════════════════════════════════
\section{The Negation Involution and the Parity Law}
\label{sec:parity}
% ═══════════════════════════════════════════════════════════════════════

\subsection{The involution}

\begin{definition}\label{def:involution}
The \emph{negation involution} on $\mathbb{Z}/p\mathbb{Z}$
is the map $\iota(x) = p - x \pmod{p}$.
On the nonzero elements, $\iota$ is a fixed-point-free involution
(since $p$ is odd, $\iota(x) = x$ implies $2x \equiv 0$,
hence $x \equiv 0$).
\end{definition}

\begin{lemma}[Spin-flip lemma]\label{lem:spinflip}
For $a \not\equiv 0 \pmod{p}$:
\[
    \Leg{p - a}{p}
    \;=\;
    \Leg{-1}{p} \cdot \Leg{a}{p}
    \;=\;
    \begin{cases}
    -\Leg{a}{p} & \text{if } p \equiv 3 \pmod{4}, \\[4pt]
    +\Leg{a}{p} & \text{if } p \equiv 1 \pmod{4}.
    \end{cases}
\]
\end{lemma}

\begin{proof}
$\Leg{p-a}{p} = \Leg{-a}{p} = \Leg{-1}{p}\Leg{a}{p}$.
By the classical evaluation, $\Leg{-1}{p} = (-1)^{(p-1)/2}$,
which is $-1$ when $p \equiv 3$ and $+1$ when $p \equiv 1$.
\end{proof}

\subsection{Involution closure of the right half}

\begin{lemma}\label{lem:closure}
Let $n \ge 2$ with $p = 2n - 1$ prime.
Let $S_R$ denote the set of residues modulo $p$ of elements
of $\Rhalf$, and let $S_R^* = S_R \setminus \{0\}$.
Then $\iota(S_R^*) = S_R^*$; that is, $S_R^*$ is closed
under the negation involution.
\end{lemma}

\begin{proof}
The elements of $\Rhalf$ have offsets $n, n+1, \ldots, 2n-2$
from $(n-1)^2$.
Since $(n-1)^2 \equiv r \pmod{p}$ (where $r = n^2 \bmod p$),
the residues of $\Rhalf$ are
$\{r + n,\; r + n + 1,\; \ldots,\; r + 2n - 2\} \pmod{p}$,
a set of $n - 1$ consecutive residue classes.

For $a = r + j$ with $n \le j \le 2n - 2$, we have
$\iota(a) = p - r - j$.
From the proof of Lemma~\ref{lem:anchor}:
$r \equiv n^2 \equiv 4^{-1} \pmod{p}$, so
$p - 2r \equiv p - 2 \cdot 4^{-1} \equiv p - (p+1)/2
\equiv (p-1)/2 = n - 1 \pmod{p}$
(when $p \equiv 3 \bmod 4$; the $p \equiv 1$ case is similar).

Thus $\iota(a) = (n - 1) - j + r \pmod{p}$,
and we need to check that $(n-1) - j$ maps
$\{n, \ldots, 2n-2\}$ to itself modulo $p$.
Indeed, as $j$ ranges from $n$ to $2n - 2$,
$(n-1) - j$ ranges from $-1$ to $-(n-1)$,
which modulo $p = 2n - 1$ is $2n - 2$ down to $n$.
This is exactly the same set.

The argument is identical for $p \equiv 1 \pmod{4}$
(using the corresponding value of $4^{-1}$).
\end{proof}

\subsection{The main theorem}

\begin{theorem}[Oppermann Parity Law]\label{thm:parity}
Let $n \ge 2$ be an \emph{even} integer with
$p = 2n - 1$ prime.
Then:
\begin{enumerate}[nosep, label=(\roman*)]
\item The right half $\Rhalf$ contains
$(n-2)/2$ quadratic residues,
$(n-2)/2$ quadratic non-residues,
and $1$ multiple of $p$:
\[
    \mathrm{Skew}_p(\Rhalf) = 0.
\]
\item The left half $\Lhalf$ contains
$(n-2)/2$ quadratic residues and $n/2$
quadratic non-residues:
\[
    \mathrm{Skew}_p(\Lhalf) = 1.
\]
\item The global skewness $N^- - N^+ = 1$ is carried
entirely by the left half.
\end{enumerate}
\end{theorem}

\begin{proof}
Since $n$ is even, $p \equiv 3 \pmod{4}$
(Lemma~\ref{lem:anchor}).

\textbf{(i)}
By Lemma~\ref{lem:anchor}(i), the anchor lies in $\Rhalf$.
The remaining $n - 2$ elements of $\Rhalf$ have nonzero
residues forming the set $S_R^*$.
By Lemma~\ref{lem:closure}, $S_R^*$ is closed under $\iota$.
Since $n - 2$ is even (as $n$ is even), $\iota$ partitions
$S_R^*$ into $(n-2)/2$ orbits of size $2$.
By Lemma~\ref{lem:spinflip}, each orbit $\{a, \iota(a)\}$
contains one QR and one NR (since $p \equiv 3 \pmod{4}$).
Therefore $N^+_R = N^-_R = (n-2)/2$ and
$\mathrm{Skew}_p(\Rhalf) = 0$.

\textbf{(ii)} Follows from
$\mathrm{Skew}_p(\Lhalf) = 1 - \mathrm{Skew}_p(\Rhalf) = 1$
via \eqref{eq:global}.

\textbf{(iii)} Immediate from (i) and (ii).
\end{proof}

\subsection{The complementary case: odd $n$}

\begin{theorem}\label{thm:odd}
Let $n \ge 3$ be an \emph{odd} integer with
$p = 2n - 1$ prime.
Then the right half skewness $\mathrm{Skew}_p(\Rhalf)$
is always \emph{even} (but generically nonzero).
\end{theorem}

\begin{proof}
When $n$ is odd, $p \equiv 1 \pmod{4}$ and the anchor
lies in $\Lhalf$ (Lemma~\ref{lem:anchor}(ii)).
Thus $\Rhalf$ has $n - 1$ elements, all with nonzero
residues.
By Lemma~\ref{lem:closure}, $S_R^*$ is still closed under
$\iota$, and since $n - 1$ is even, $\iota$ partitions
$S_R^*$ into $(n-1)/2$ pairs.
But now $p \equiv 1 \pmod{4}$, so by
Lemma~\ref{lem:spinflip}, $\iota$ \emph{preserves} the
Legendre symbol: each pair $\{a, \iota(a)\}$ consists of
two QRs or two NRs.
Hence $N^+_R$ and $N^-_R$ are both even, and
$\mathrm{Skew}_p(\Rhalf) = N^-_R - N^+_R$ is even.
\end{proof}

\begin{remark}\label{rem:contrast}
The contrast between even and odd $n$ is entirely
controlled by the single arithmetic invariant $\Leg{-1}{p}$:
\begin{center}
\renewcommand{\arraystretch}{1.2}
\begin{tabular}{c c c c}
\toprule
$n$ parity & $p \bmod 4$ & $\Leg{-1}{p}$ &
$\iota$ action on QR/NR \\
\midrule
even & $3$ & $-1$ & swaps QR $\leftrightarrow$ NR
$\;\Rightarrow\;$ pairing $\;\Rightarrow\;$ skewness $= 0$ \\
odd  & $1$ & $+1$ & preserves QR, NR
$\;\Rightarrow\;$ no cross-pairing $\;\Rightarrow\;$ skewness even \\
\bottomrule
\end{tabular}
\end{center}
\end{remark}

% ═══════════════════════════════════════════════════════════════════════
\section{Computational Verification}\label{sec:computation}
% ═══════════════════════════════════════════════════════════════════════

\subsection{Methodology}

For each $n$ with $p = 2n - 1$ prime, the scanner
computes the Jacobi symbol $\Leg{x}{p}$ for every integer $x$
in each half-interval, tallies positive/negative/zero spins,
and records the skewness $N^- - N^+$ for each half.

\subsection{Results for even $n$}

\begin{table}[H]
\centering
\caption{Half-interval spin distribution for even $n$
($p \equiv 3 \pmod{4}$).
In every case, $\mathrm{Skew}(\Rhalf) = 0$ and the
anchor lies in $\Rhalf$.}
\label{tab:even}
\medskip
\begin{tabular}{@{}r r r r r r r r r@{}}
\toprule
$n$ & $p$ &
$N^+_L$ & $N^-_L$ & $N^0_L$ & Skew$_L$ &
$N^+_R$ & $N^-_R$ & Skew$_R$ \\
\midrule
2   & 3   & 0  & 1  & 0  & 1  & 0  & 0  & \textbf{0} \\
4   & 7   & 1  & 2  & 0  & 1  & 1  & 1  & \textbf{0} \\
6   & 11  & 2  & 3  & 0  & 1  & 2  & 2  & \textbf{0} \\
10  & 19  & 4  & 5  & 0  & 1  & 4  & 4  & \textbf{0} \\
12  & 23  & 5  & 6  & 0  & 1  & 5  & 5  & \textbf{0} \\
16  & 31  & 7  & 8  & 0  & 1  & 7  & 7  & \textbf{0} \\
22  & 43  & 10 & 11 & 0  & 1  & 10 & 10 & \textbf{0} \\
100 & 199 & 49 & 50 & 0  & 1  & 49 & 49 & \textbf{0} \\
\bottomrule
\end{tabular}
\end{table}

\begin{proposition}\label{prop:verified}
Theorem~\ref{thm:parity} has been verified computationally
for all 87 even values of $n \le 500$ with $2n - 1$ prime.
In every case, $\mathrm{Skew}_p(\Rhalf) = 0$ and
$\mathrm{Skew}_p(\Lhalf) = 1$.
\end{proposition}

\subsection{Results for odd $n$}

\begin{table}[H]
\centering
\caption{Half-interval spin distribution for odd $n$
($p \equiv 1 \pmod{4}$).
The right half skewness is always even but varies.
The anchor lies in $\Lhalf$.}
\label{tab:odd}
\medskip
\begin{tabular}{@{}r r r r r r r r r@{}}
\toprule
$n$ & $p$ &
$N^+_L$ & $N^-_L$ & $N^0_L$ & Skew$_L$ &
$N^+_R$ & $N^-_R$ & Skew$_R$ \\
\midrule
3   & 5   & 1  & 0  & 1  & $-1$ & 0  & 2  & $2$ \\
7   & 13  & 3  & 2  & 1  & $-1$ & 2  & 4  & $2$ \\
9   & 17  & 5  & 2  & 1  & $-3$ & 2  & 6  & $4$ \\
15  & 29  & 9  & 4  & 1  & $-5$ & 4  & 10 & $6$ \\
21  & 41  & 13 & 6  & 1  & $-7$ & 6  & 14 & $8$ \\
\bottomrule
\end{tabular}
\end{table}

% ═══════════════════════════════════════════════════════════════════════
\section{Structural Interpretation}\label{sec:interpretation}
% ═══════════════════════════════════════════════════════════════════════

\subsection{The right half as an algebraically neutral zone}

For even $n$, the right half $\Rhalf$ is in a state of
\emph{perfect algebraic equilibrium}: its quadratic residue
distribution is exactly symmetric.
This is not a statistical accident; it is forced by the
involution $\iota$ acting as a QR/NR swap on a closed
domain.

The left half, by contrast, carries the full topological
defect: it is the sole repository of the global skewness
$+1$.
This asymmetry between the halves echoes the asymmetry
in Oppermann's conjecture itself, which treats the left
and right half-intervals as structurally distinct.

\subsection{Obstruction interpretation}

A hypothetical ``prime vacuum'' in $\Rhalf$
(all elements composite) would require $n - 1$ consecutive
composites maintaining \emph{exact} QR/NR balance modulo $p$.
While the Parity Law proves this balance is algebraically
mandated (regardless of whether primes are present),
the \emph{multiplicative} realization of such a balanced
configuration by composites alone faces additional
constraints not captured by the residue structure.

We emphasize that the Parity Law does not by itself
prove Oppermann's conjecture.
It establishes a structural framework---the algebraic
neutrality of $\Rhalf$ and the defect-carrying role of
$\Lhalf$---within which the conjecture can be studied.

\subsection{Limitations}

The Parity Law applies only when $p = 2n - 1$ is prime,
which restricts to roughly half of all $n$ values.
When $p$ is composite, the Jacobi symbol replaces the
Legendre symbol, the QR/NR counts are no longer simply
$(p-1)/2$ each, and the involution argument does not
directly apply.
Extending the framework to composite $p$ is an open problem.

% ═══════════════════════════════════════════════════════════════════════
\section{Discussion}\label{sec:discussion}
% ═══════════════════════════════════════════════════════════════════════

\subsection{Relation to character sums}

The Parity Law can be restated as a character sum identity.
For even $n$ with $p = 2n - 1$ prime:
\[
    \sum_{x \in \Rhalf} \Leg{x}{p}
    \;=\; N^+_R - N^-_R
    \;=\; 0.
\]
This is a \emph{partial character sum} over $n - 1$
consecutive integers.
While partial character sums are typically bounded by
$O(\sqrt{p}\log p)$ via the P\'olya--Vinogradov inequality,
here we obtain an \emph{exact} evaluation of zero---a much
stronger result, albeit for a specific sub-interval
determined by the arithmetic of $n$.

\subsection{Connection to Paper IX}

The Spin Asymmetry Theorem of \cite{ChenSpin}
established the global identity $N^- - N^+ = 1$.
The present paper decomposes this identity into
local contributions:
\[
    \underbrace{1}_{\text{left}} \;+\;
    \underbrace{0}_{\text{right}} \;=\; 1
    \qquad\text{(even $n$)},
    \qquad\qquad
    \underbrace{1 - R}_{\text{left}} \;+\;
    \underbrace{R}_{\text{right}} \;=\; 1
    \qquad\text{(odd $n$, $R$ even)}.
\]
The even case is maximally rigid; the odd case
distributes the skewness in a manner that remains
to be fully understood.

\subsection{Open questions}

\begin{enumerate}[label=(\arabic*)]
\item \emph{Odd $n$ skewness.}
For odd $n$, the right half skewness
$\mathrm{Skew}_p(\Rhalf)$ is always a positive even
integer in our data. Is it always positive?
Is there a closed-form expression in terms of class
numbers or character sums?

\item \emph{Sub-interval character sums.}
Can the exact vanishing of the partial character sum
over $\Rhalf$ be derived from classical results on
Gauss sums or Jacobi sums?

\item \emph{Composite lengths.}
When $p = 2n - 1$ is composite, can a modified involution
argument yield analogous (approximate) parity constraints?

\item \emph{Multiplicative refinement.}
Among the $(n-2)/2$ QR--NR pairs in $\Rhalf$ (for even $n$),
how many pairs consist of a prime and a composite?
This would connect the algebraic structure to the
distribution of primes in the half-interval.
\end{enumerate}

\subsection*{Data and code availability}

Source code and the \LaTeX{} manuscript are available at
\url{https://github.com/Ruqing1963/oppermann-parity-law}.
The companion Paper~IX repository is at
\url{https://github.com/Ruqing1963/legendre-spin-asymmetry}.

% ═══════════════════════════════════════════════════════════════════════
\begin{thebibliography}{99}

\bibitem{ChenConductor}
  R.~Chen,
  ``Conductor incompressibility for Frey curves associated to
  prime gaps,''
  Zenodo, 2026.
  \url{https://zenodo.org/records/18682375}

\bibitem{ChenLandau}
  R.~Chen,
  ``On Landau's fourth problem: conductor rigidity and
  Sato--Tate equidistribution for the $n^2+1$ family,''
  Zenodo, 2026.
  \url{https://zenodo.org/records/18683712}

\bibitem{ChenAP}
  R.~Chen,
  ``The 2-2 coincidence: conductor rigidity for primes in
  arithmetic progressions and the Bombieri--Vinogradov barrier,''
  Zenodo, 2026.
  \url{https://zenodo.org/records/18684151}

\bibitem{ChenFF}
  R.~Chen,
  ``The geometry of prime vacuums: Legendre's conjecture in
  function fields via monodromy and Chebotarev density,''
  Zenodo, 2026.
  \url{https://zenodo.org/records/18705744}

\bibitem{ChenSpin}
  R.~Chen,
  ``Algebraic rigidity and quadratic residue asymmetry in
  Legendre intervals,''
  Zenodo, 2026.
  \url{https://zenodo.org/records/18706876}

\end{thebibliography}

\end{document}
